%%
%% Advanced NLP Exercise 2.1 - Course Report
%% Based on ACM sigconf template
%%
\documentclass[sigconf]{acmart}

\usepackage{booktabs}

%% Remove ACM-specific metadata for course report
\settopmatter{printacmref=false}
\renewcommand\footnotetextcopyrightpermission[1]{}
\pagestyle{plain}
\setcopyright{none}
\acmConference{}{}{}
\acmBooktitle{}
\raggedbottom

%% Figure path - on Overleaf put PNGs in a figures/ folder
\graphicspath{{figures/}}

\bibliographystyle{ACM-Reference-Format}

\begin{document}

\title{Multilingual ADE Detection and Classical ML vs.\ LLM Comparison}

\author{Hatice Kübra Aksu}
\affiliation{%
  \institution{Technische Universit\"at Berlin}
  \city{Berlin}
  \country{Germany}
}

\begin{abstract}
This report covers two tasks.
In Task~1 I fine-tune XLM-RoBERTa-base for adverse drug event detection on German and English social media data from SMM4H~2026, then test zero-shot transfer to Russian.
I compare monolingual, multilingual, and machine-translation--augmented setups.
In Task~2 I compare a Random Forest with DistilBERT (on textified features) and Flan-T5 few-shot prompting on the Iris dataset.
Main takeaways: multilingual pretraining enables decent cross-lingual transfer even across scripts, but dataset size matters a lot.
On tabular data, classical ML wins easily.
\end{abstract}

\maketitle
\pagestyle{plain}  % ACM class resets pagestyle in \maketitle; re-apply here

%% ============================================================
\section{Introduction}
\label{sec:intro}

Multilingual pretrained models like XLM-RoBERTa~\cite{conneau2020xlmr} can share learned representations across languages.
This is useful in the medical domain where labelled data is scarce for most languages.
Detecting adverse drug events (ADEs) in social media is a well-studied problem from the SMM4H shared task series~\cite{klein2025smm4h}.

The idea behind cross-lingual transfer is straightforward: if a model learns what an adverse reaction looks like in English, it should be able to recognize one in German or Russian too, even without training data in those languages.
XLM-RoBERTa was pretrained on 100 languages using CommonCrawl data, so it has seen all three scripts we work with here.
The question is how much of that actually carries over to a specific downstream task like ADE detection.

In Task~1 (Section~\ref{sec:task1}) I train ADE classifiers on German and English, then test zero-shot on Russian, which uses Cyrillic script and makes this a real cross-script challenge.
I also try machine-translating English training data to Russian to see if that helps.

In Task~2 (Section~\ref{sec:task2}) I take a different angle: comparing classical ML against transformers and LLMs on the Iris dataset~\cite{fisher1936iris}, after converting numerical features to text.
The point is to check whether language models can handle structured, numerical data or whether classical approaches are still the better tool for that.


%% ============================================================
\section{Task 1: Multilingual ADE Detection}
\label{sec:task1}

\subsection{Data Preparation (Task 1.1)}
\label{sec:task1-data}

I use the SMM4H~2026 Task~1 dataset.
It has social media posts labelled for adverse drug events (binary: ADE or not).
The sources are quite different: German posts come from lifeline.de (a health forum), English from Twitter, and Russian from the RuDReC drug review corpus.

I picked three languages: German (DE) and English (EN) for training, Russian (RU) for zero-shot evaluation.
This gives a nice setup: two training languages with Latin script, and a test language with Cyrillic script that the model never sees during training.

The provided dev sets become my test sets.
I split off 15\% from each training set (stratified) as validation.
Table~\ref{tab:task1-splits} shows the sizes.

\begin{table}[t]
\centering
\caption{Dataset sizes. RU is test-only (zero-shot).}
\label{tab:task1-splits}
\small
\begin{tabular}{llrrr}
\toprule
\textbf{Lang} & \textbf{Split} & \textbf{Total} & \textbf{Pos} & \textbf{Pos\%} \\
\midrule
DE & train & 1{,}070 & 60  & 5.6  \\
DE & val   & 189    & 11  & 5.8  \\
DE & test  & 1{,}259 & 70  & 5.6  \\
\midrule
EN & train & 15{,}553 & 1{,}666 & 10.7 \\
EN & val   & 2{,}745  & 294   & 10.7 \\
EN & test  & 444     & 61    & 13.7 \\
\midrule
RU & test  & 3{,}783  & 544   & 14.4 \\
\midrule
\multicolumn{2}{l}{EN$\to$RU translated} & 500 & 100 & 20.0 \\
\bottomrule
\end{tabular}
\end{table}

The class imbalance is significant.
Only 5.6\% of German training samples are positive.
English is better at 10.7\% but still heavily skewed.
This matters a lot later because the German monolingual model essentially can't learn from 60 positive examples.

For Task~1.3.3 I translated 500 English samples to Russian using Google Translate.
I oversampled positives to 20\% in the translated set (100 out of 500) to offset the class imbalance a bit.
The idea was that giving the model more positive examples in the target language might improve recall.

\subsection{Exploratory Analysis (Task 1.2)}
\label{sec:task1-eda}

\begin{figure}[t]
\centering
\includegraphics[width=\linewidth]{label_distribution.png}
\caption{Label distribution. All three languages are dominated by the negative class.}
\label{fig:label-dist}
\end{figure}

Figure~\ref{fig:label-dist} shows the label distribution.
German is the worst at $\sim$5.6\% positive.
English and Russian sit around 10--14\%.
The imbalance is a challenge for all models, but especially for German where the positive class is barely represented.

Text lengths vary a lot across languages, which reflects the different platforms each corpus comes from.
German forum posts tend to be long, often spanning multiple paragraphs where people describe symptoms in detail and mention several medications and side effects.
English tweets are short (Twitter character limit).
Russian reviews are somewhere in between: more detailed than tweets, but more focused than open-ended forum discussions.

\begin{figure*}[t]
\centering
\includegraphics[width=\textwidth]{token_length_distribution.png}
\caption{Token counts (XLM-RoBERTa tokenizer). Red line = 128-token limit. Most EN and RU texts fit; some DE posts get truncated.}
\label{fig:token-dist}
\end{figure*}

Figure~\ref{fig:token-dist} shows token lengths after XLM-RoBERTa tokenization.
English and Russian mostly stay under 128 tokens, but some German posts go well beyond that and get cut off.
I kept the 128 limit anyway to save training time, which probably hurts German performance.

I also looked at word frequencies.
Each language has its own medical vocabulary (drug names, symptoms, body parts), and there's almost no surface-level overlap between them.
German uses words like ``Nebenwirkungen'' (side effects) and ``Kopfschmerzen'' (headache), which look nothing like their English counterparts.
Russian is even further away because of the Cyrillic script.
So XLM-RoBERTa has to do all the heavy lifting, mapping words that look completely different into similar internal representations just because they mean the same thing.

I also manually inspected a sample of the EN$\to$RU translations.
Common drug names like ``ibuprofen'' or ``metformin'' survive translation fine, but informal descriptions of side effects often come out stilted or overly literal.
For example, English tweets that say things like ``this med makes me so dizzy'' get translated into grammatically correct but unnatural-sounding Russian that doesn't match how native speakers would describe the same symptom.
A few translations also dropped or altered negations, which could flip the meaning entirely.

\subsection{Model Training (Task 1.3)}
\label{sec:task1-training}

All experiments use \texttt{xlm-roberta-base} with the same hyperparameters: learning rate $2 \times 10^{-5}$, batch size~16, 5~epochs, max length~128, AdamW with linear warmup (10\%), and weighted cross-entropy for the class imbalance.
I keep the best checkpoint by validation macro-F1.
The class weights are computed using sklearn's \texttt{compute\_class\_weight("balanced")}.

\subsubsection{Monolingual Models (Task 1.3.1)}

Two monolingual models:
\begin{itemize}
\item \textbf{Mono-DE}: Trained on German only.
1{,}070 samples with just 60 positives.
The model couldn't learn the minority class at all and just predicts everything as negative.
Macro-F1 = 0.486, which is basically the negative-class F1 averaged with zero.
Not useful, but not surprising either with this little data.
Even with weighted loss, 60 positive examples aren't enough for a 278M-parameter transformer to learn anything meaningful.
\item \textbf{Mono-EN}: Trained on English only.
15{,}553 samples is enough to learn well.
Macro-F1 = 0.869 on English test, with 0.704 precision and 0.820 recall on the positive class.
\end{itemize}

\subsubsection{Multilingual Model (Task 1.3.2)}

I combined DE and EN training data into one model (16{,}623 training samples total).
Results are in Table~\ref{tab:task1-results}.
The interesting part: it reaches 0.689 macro-F1 on Russian without ever seeing a single Russian training sample.
XLM-RoBERTa's cross-lingual representations actually work across the Latin--Cyrillic boundary.

On the other hand, the multilingual model does slightly worse on English compared to the English-only model (0.796 vs.\ 0.869).
This trade-off makes sense: the model now has to handle two languages at once.
The small, noisy German data drags the model away from the optimal English performance.

\subsubsection{Translation-Based Transfer (Task 1.3.3)}

Three ways to evaluate on Russian:
\begin{enumerate}
\item[(a)] English-only model $\to$ Russian test (zero-shot)
\item[(b)] Multilingual DE+EN model $\to$ Russian test (zero-shot)
\item[(c)] Model trained on 500 machine-translated EN$\to$RU samples $\to$ Russian test
\end{enumerate}

\begin{figure}[t]
\centering
\includegraphics[width=\linewidth]{task133_ru_comparison.png}
\caption{Task 1.3.3: macro-F1 on Russian. EN-only wins; translation doesn't help.}
\label{fig:task133}
\end{figure}

Figure~\ref{fig:task133} shows the comparison.
The English-only model does best (roughly 0.70 macro-F1), slightly beating the multilingual model (0.689).
The translated model is worst at 0.671.

500 translated samples just can't compete with a model trained on 15{,}553 English samples.
The EN-only model has seen far more diverse ADE examples, and XLM-RoBERTa's cross-lingual alignment transfers that knowledge to Russian even without explicit Russian data.
Machine translation also adds noise. Medical terms don't always translate cleanly, and informal tweet language tends to produce awkward translations.

\subsection{Evaluation and Error Analysis (Task 1.4)}
\label{sec:task1-eval}

\begin{table}[t]
\centering
\caption{All Task~1 results. Precision and recall are for the positive (ADE) class.}
\label{tab:task1-results}
\small
\begin{tabular}{llcccc}
\toprule
\textbf{Model} & \textbf{Test} & \textbf{M-F1} & \textbf{Prec} & \textbf{Rec} & \textbf{AUC} \\
\midrule
Mono DE       & DE & .486 & .000 & .000 & .280 \\
Mono EN       & EN & .869 & .704 & .820 & .976 \\
\midrule
Multi DE+EN   & DE & .603 & .267 & .229 & .819 \\
Multi DE+EN   & EN & .796 & .550 & .721 & .961 \\
Multi DE+EN   & RU & .689 & .465 & .415 & .830 \\
\midrule
EN-only (a)   & RU & .700 & .541 & .390 & .839 \\
Multi (b)     & RU & .689 & .465 & .415 & .830 \\
Translated (c)& RU & .671 & .357 & .522 & .789 \\
\bottomrule
\end{tabular}
\end{table}

Table~\ref{tab:task1-results} has all the numbers.
Here's what I found:

\textbf{German mono model} fails completely: zero precision, zero recall on positives.
60 positive samples just isn't enough for a transformer to learn from.
The model just predicts everything as negative, which gives 94.5\% accuracy (because the data is 94.4\% negative) but doesn't help with the actual task.

\textbf{Multilingual training helps German}, jumping from 0.486 to 0.603.
The English data teaches the model what ADEs look like, and that transfers to German through XLM-RoBERTa's shared representations.
Still not great (only 22.9\% recall), but the model at least starts detecting some positives.

\textbf{Zero-shot Russian works} better than I expected.
All models get AUC-ROC above 0.78, meaning the models rank positive and negative examples reasonably well even across scripts.
The AUC-ROC is much higher than the macro-F1, which suggests the model is learning useful patterns but the default 0.5 threshold isn't optimal. Tuning it could improve F1.
EN-only edges out the multilingual model, probably because the small German data adds more noise than signal.

\textbf{Translation hurts more than it helps}.
The translated model has highest recall (0.522) but lowest precision (0.357).
The noisy translations make it overpredict ADEs. The model seems to pick up on translation artifacts rather than actual ADE patterns.
With only 500 training samples, it also can't match the variety of the full English training set.

\begin{figure*}[t]
\centering
\includegraphics[width=\textwidth]{confusion_matrices_ru.png}
\caption{Confusion matrices on Russian. Main problem: too many false negatives across all models.}
\label{fig:cm-ru}
\end{figure*}

Figure~\ref{fig:cm-ru} shows confusion matrices for Russian.
The main error pattern across all three models is false negatives, i.e.\ missed ADEs.
The EN-only and multilingual models are conservative: they get decent precision but miss many true positives.
The translated model flips this: it catches more ADEs but also flags many non-ADE posts as positive.

Looking at the false negatives more closely, many of them are posts where the adverse effect is described indirectly or uses informal language.
The models trained on English tweets and German forum posts struggle with colloquial Russian expressions for symptoms.
False positives tend to be posts that mention drugs and negative experiences but not actual side effects, like complaints about drug prices or availability.


%% ============================================================
\section{Task 2: Classical ML vs.\ LLMs on Iris}
\label{sec:task2}

\subsection{Random Forest Baseline (Task 2.1)}
\label{sec:task2-rf}

I use the Iris dataset~\cite{fisher1936iris} with a 60/20/20 stratified split (90/30/30 samples).
A Random Forest with 100 trees and max depth 5~\cite{breiman2001rf} gets 93.3\% accuracy and macro-F1 on test.
I also set \texttt{class\_weight="balanced"} for consistency with the Task~1 approach, though it doesn't matter much here since Iris has perfectly balanced classes.

\begin{figure}[t]
\centering
\includegraphics[width=0.85\linewidth]{rf_feature_importance.png}
\caption{Feature importance. Petal measurements dominate, as expected.}
\label{fig:rf-importance}
\end{figure}

Figure~\ref{fig:rf-importance} shows feature importances.
Petal length and width are by far the most useful, together accounting for over 80\% of the total importance.
The decision rules are simple: petal length $<$ 2.5 separates setosa from the other two, petal width around 1.75 separates versicolor from virginica.
Sepal measurements contribute relatively little.
This is consistent with the well-known structure of the dataset: setosa is linearly separable, while versicolor and virginica overlap slightly.

The two misclassified test samples are both from the versicolor--virginica boundary.
Their petal measurements fall right in the overlap region where even human experts would have trouble.
93.3\% is essentially the ceiling for this particular split.

\subsection{Textification and DistilBERT (Task 2.2)}
\label{sec:task2-text}

The idea here is to check whether a transformer can learn to classify structured data if we first convert it to text.
I convert each sample in two ways:
\begin{enumerate}
\item \textbf{Natural language}: ``This flower has a sepal length of 5.1\,cm, a sepal width of 3.5\,cm, a petal length of 1.4\,cm, and a petal width of 0.2\,cm.''
\item \textbf{Structured}: \texttt{sepal\_length=5.1 | sepal\_width=3.5 | petal\_length=1.4 | petal\_width=0.2}
\end{enumerate}

Then I fine-tune DistilBERT~\cite{sanh2019distilbert} for 20 epochs (batch size 8, lr $3\times10^{-5}$, max length 64).
The structured format matches the RF exactly at 93.3\%.
Natural language gets 90.0\%.

The structured format works better because it's more compact and consistent.
Every sample follows the same pattern: \texttt{feature=value | feature=value | ...}.
The model can focus on the numbers right after each feature name without getting distracted by filler words.
The natural language version has extra words (``has a'', ``of'', ``centimeters'') that the model needs to learn to ignore, and with only 90 training samples that's harder to do.

So transformers can learn from textified numbers, but they don't beat the RF baseline.
With 20 epochs on 90 samples, the model is probably overfitting to the training set.
The fact that it still matches the RF on structured data shows that the text representation isn't the bottleneck; the limited training size is.

\subsection{LLM Few-Shot and Hybrid (Task 2.3)}
\label{sec:task2-llm}

I use Flan-T5-large~\cite{chung2022flan} with 2 examples per class (6 examples total in the prompt).
I pick examples closest to each class centroid, so the model sees the most typical representative of each species.
Two setups:
\begin{itemize}
\item \textbf{Basic few-shot}: natural language feature descriptions plus examples.
The prompt asks the model to classify and respond with one word.
\item \textbf{Augmented}: same few-shot examples, but I also feed in the RF's prediction with confidence score and simplified decision rules (e.g., ``petal\_length$<$2.5 $\to$ setosa'').
The idea was that the LLM could use the RF output as a strong hint.
\end{itemize}

Basic few-shot gets only 43.3\% accuracy (macro-F1 = 0.349), and the augmented version drops to 33.3\% (macro-F1 = 0.167).
The LLM just can't do the precise numerical comparisons that Iris needs.
It understands the general concept of flower classification, but deciding whether 1.7 is above or below 1.75 requires exact arithmetic, which is not what language models are built for.

The augmented version does even worse, which I didn't expect.
I think the extra information in the prompt (decision rules, confidence scores, RF predictions) ends up confusing the model more than helping it.
A simpler prompt with just the raw features seems to work better.

\subsection{Comparison (Task 2.4)}
\label{sec:task2-compare}

\begin{table}[t]
\centering
\caption{All Task~2 results on the Iris test set (30 samples).}
\label{tab:task2-results}
\small
\begin{tabular}{lcc}
\toprule
\textbf{Approach} & \textbf{Accuracy} & \textbf{Macro-F1} \\
\midrule
Random Forest           & 0.933 & 0.933 \\
DistilBERT (NL)         & 0.900 & 0.900 \\
DistilBERT (Structured) & 0.933 & 0.933 \\
LLM Few-shot            & 0.433 & 0.349 \\
LLM Augmented           & 0.333 & 0.167 \\
\bottomrule
\end{tabular}
\end{table}

\begin{figure}[t]
\centering
\includegraphics[width=\linewidth]{task2_comparison.png}
\caption{All five approaches. RF and fine-tuned DistilBERT work; LLM few-shot doesn't.}
\label{fig:task2-compare}
\end{figure}

Table~\ref{tab:task2-results} and Figure~\ref{fig:task2-compare} show the results.
Random Forest is the best or tied-best, and DistilBERT with structured text matches it exactly. So fine-tuning on textified data actually works when the format is consistent.
The LLM approaches fail badly.
Iris classification needs exact thresholds (``is petal width above 1.75?''), and that's just not what language models are built for.

Also, there's a practical side worth mentioning: the RF trains in under a second, DistilBERT takes several minutes, and Flan-T5 inference takes longer still.
For small, clean tabular data, classical ML wins by a large margin and is also the simplest option.
In terms of interpretability, RF is the clear winner: feature importances and decision rules are directly readable.
DistilBERT is a black box, though its predictions are consistent.
The LLM is the least reliable: its outputs vary and it sometimes predicts classes not in the label set.
Honestly, Iris is probably too easy and too clean for this kind of comparison. It would be more interesting to try a messier, higher-dimensional dataset where the decision boundaries aren't so obvious.


%% ============================================================
\section{Conclusion}
\label{sec:conclusion}

\textbf{Task~1}: XLM-RoBERTa enables cross-lingual ADE detection even across the Latin--Cyrillic barrier.
The English-only model got the best zero-shot Russian performance (macro-F1 = 0.700).
I expected the translation approach to help more than it did --- but 500 translated samples just can't match 15k English ones, and translation noise makes things worse.
The biggest limitation was the tiny German dataset: 60 positive samples is not enough for a transformer.
With more balanced German data, the multilingual model would likely improve on Russian too.

\textbf{Task~2}: Random Forest (93.3\%) beats everything on Iris.
Fine-tuned DistilBERT can match it with structured text input, but few-shot Flan-T5 fails at numerical reasoning.
For tabular data, classical ML is the way to go.
Language models are designed for language, not for numerical comparisons. Converting numbers to text and hoping the model can do arithmetic doesn't work.

\textbf{Limitations and future work}: For Task~1, the most obvious improvement would be a larger, more balanced German corpus.
A higher max token length (256 or 512) could also help for the longer German forum posts that get truncated at 128.
For the translation experiment, translating the full English training set instead of just 500 samples might change the picture, especially with a medical-domain translation model instead of Google Translate.
For Task~2, testing with a harder tabular dataset (more features, more classes, noisier data) would give a clearer picture of where the break-even point between classical ML and transformers lies.

\bibliography{refs}

\end{document}
